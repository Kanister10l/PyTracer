\documentclass[a4paper,12pt,twoside,openany]{report}
%
% Wzorzec pracy dyplomowej
% J. Starzynski (jstar@iem.pw.edu.pl) na podstawie pracy dyplomowej
% mgr. inż. Błażeja Wincenciaka
% Wersja 0.1 - 8 października 2016
%
\usepackage{polski}
\usepackage{helvet}
\usepackage[T1]{fontenc}
\usepackage{anyfontsize}
\usepackage[utf8]{inputenc}
\usepackage[pdftex]{graphicx}
\usepackage{tabularx}
\usepackage{array}
\usepackage[polish]{babel}
\usepackage{subfigure}
\usepackage{amsfonts}
\usepackage{verbatim}
\usepackage{indentfirst}
\usepackage[pdftex]{hyperref}


% rozmaite polecenia pomocnicze
% gdzie rysunki?
\newcommand{\ImgPath}{.}

% oznaczenie rzeczy do zrobienia/poprawienia
\newcommand{\TODO}{\textbf{TODO}}


% wyroznienie slow kluczowych
\newcommand{\tech}{\texttt}

% na oprawe (1.0cm - 0.7cm)*2 = 0.6cm
% na oprawe (1.1cm - 0.7cm)*2 = 0.8cm
%  oddsidemargin lewy margines na nieparzystych stronach
% evensidemargin lewy margines na parzystych stronach
\def\oprawa{1.05cm}
\addtolength{\oddsidemargin}{\oprawa}
\addtolength{\evensidemargin}{-\oprawa}

% table span multirows
\usepackage{multirow}
\usepackage{enumitem}	% enumitem.pdf
\setlist{listparindent=\parindent, parsep=\parskip} % potrzebuje enumitem

%%%%%%%%%%%%%%% Dodatkowe Pakiety %%%%%%%%%%%%%%%%%
\usepackage{prmag2017}   % definiuje komendy opieku,nrindeksu, rodzaj pracy, ...


%%%%%%%%%%%%%%% Strona Tytułowa %%%%%%%%%%%%%%%%%
% To trzeba wypelnic swoimi danymi
\title{Opracowanie algorytmu do wyznaczania ścieżki sondy w pomiarach trójwymiarowych obiektów}

% autor
\author{Kamil Michał Samul}
\nrindeksu{284931}

% jeśli wykonawca jest tylko jeden, to usuwamy poniższe polecenia

\opiekun{dr inż. Bartosz Chaber}
%\konsultant{prof. Dzielny Konsultant}  % opcjonalnie
\terminwykonania{1 lutego 2020} % data na oświadczeniu o samodzielności
\rok{2020}


% Podziekowanie - opcjonalne
\podziekowania{\input{podziekowania.tex}}

% To sa domyslne wartosci
% - mozna je zmienic, jesli praca jest pisana gdzie indziej niz w ZETiIS
% - mozna je wyrzucic jesli praca jest pisana w ZETiIS
%\miasto{Warszawa}
%\uczelnia{POLITECHNIKA WARSZAWSKA}
%\wydzial{WYDZIAŁ ELEKTRYCZNY}
%\instytut{INSTYTUT ELEKTROTECHNIKI TEORETYCZNEJ\linebreak[1] I~SYSTEMÓW INFORMACYJNO-POMIAROWYCH}
% \zaklad{ZAKŁAD ELEKTROTECHNIKI TEORETYCZNEJ\linebreak[1] I~INFORMATYKI STOSOWANEJ}
%\kierunekstudiow{INFORMATYKA}

% domyslnie praca jest inzynierska, ale po odkomentowaniu ponizszej linii zrobi sie magisterska
%\pracamagisterska
%%% koniec od P.W

\opinie{%
  \input{opiniaopiekuna.tex}
  \newpage
  \input{recenzja.tex}
}

\streszczenia{
  \input{streszczenia.tex}
}

\begin{document}
\maketitle

%-----------------
% Wstęp
%-----------------
\chapter{Wstęp}

  \section{Cel i zakres pracy}
    \subsection{Cel pracy}
      Celem pracy jest opracowanie algorytmu wyznaczającego ścieżkę sondy pola bliskiego, 
      która pozwoli na pomiar obiektów trójwymiarowych o zróżnicowanej geometrii.
      Aktualne rozwiązanie pozwala jedynie na badanie obiektów będących jedynie płaską 
      płytką bez dodatkowych elementów zmieniających wysokość obiektu.
      Problemem, który staram się rozwiązać jest 
      \begin{itemize}
        \item zniesienie wymogu kształtu badanego obiektu
      \end{itemize}
      W przypadku pomiaru pola bliskiego elementy badane muszą być włączone w ich normalnym stanie
      podczas całego procesu ich badania. Z tego powodu usunięcie elementów, które zmieniają geometrię
      badanego obiektu w celu zachowania tradycyjnej ścieżki sondy jest nie możliwe.
      Takie ograniczenie bardzo mocno zawęża użyteczność samego skanera, co staram się rozwiązać poprzez analizę
      trójwymiarowego modelu i na jego podstawie wyznaczyć ścieżkę sondy tak, by mogła przeprowadzić pomiary
      na zdefiniowanej przez użytkownika odległości od obiektu.

    \subsection{Zakres pracy}
      Zakres pracy obejmuje analizę postawionego problemu oraz dobór odpowiednich technologii pozwalających 
      na optymalne (pod względem czasu realizacji oraz wydajności algorytmu) rozwiązanie problemu.
      \\
      Po dobraniu technologii w której zaimplementowany zostanie algorytm, należy zaplanować 
      następujące komponenty rozwiązania:
      \begin{itemize}
        \item analiza i transformacja obiektu 3D opisującego badany przedmiot
        \item wyznaczanie ścieżki na podstawie otrzymanego w poprzednim kroku obiektu 3D
        \item translacja otrzymanego rozwiązania na język zrozumiały przez skaner
        \item (opcjonalnie) wizualizacja poszczególnych etapów działania algorytmu
      \end{itemize}
      Po zakończeniu procesu tworzenia algorytmu, należy przeprowadzić fazę testów oraz jako podsumowanie, 
      przeanalizować otrzymane rezultaty wraz z wyciągnięciem z nich wniosków.
  
  \section{Opis Architektury}
    Omawiany algorytm posiada bardzo wyraźne komponenty. Biorąc pod uwagę wymagania 
    zdecydowałem się na wydzielenie z całości trzech elementów. 
    \begin{itemize}
      \item Analiza danych
      \item Logika algorytmu
      \item Zapis wyników
    \end{itemize}
    Podział problemu na pod problemy, które są mniej obszerne pod względem czasu wymaganego na analizę, 
    stworzenie kodu oraz testy, pozwala mi na ułatwienie procesu tworzenia oprogramowania.
    Mając do dyspozycji trzy samodzielnie funkcjonujące elementy aplikacji, jestem w stanie 
    w końcowej fazie projektu połączyć je ze sobą mając pewność, że każde z osobna poprawnie 
    wykonuje swoje zadanie. Dodatkowo takie założenie "modularnej" budowy pozwala mi przyjąć, 
    że w przypadku zmieniających się wymagań podczas rozwoju aplikacji, będę mógł w sposób 
    wymagający minimalnego nakładu pracy dodać lub zmodyfikować komponenty tak, by zapewnić nową funkcjonalność.

    \begin{figure}[!htbp]
      \begin{center}
    \centering
    \includegraphics[scale=0.4]{\ImgPath/rys/architektura.png}
    \end{center}
      \caption{Schemat architektury algorytmu}
      \label{schematArchitektury}
    \end{figure}

\chapter{Technologie i narzędzia}
  \section{Język programowania}
    Python jest niezwykle popularnym językiem wysokiego poziomu.
    Jest to język interpretowany. Oznacza to, że kod nie wymaga procesu kompilacji 
    do działania lecz każda jego instrukcja jest na bieżąco wykonywana przez wirtualną maszynę (pvm).
    Zdecydowałem się na ten język programowania ze względu na szerokie wsparcie w postaci 
    zewnętrznych bibliotek oraz możliwość przenoszenia kodu bez potrzeby ponownej jego kompilacji 
    w przypadku zmiany architektury systemu operacyjnego. Pierwszy powód zawdzięczam ogromnej 
    popularności Python-a oraz bardzo aktywnej społeczności, która udziela się w rozwój środowiska.
    Drugi z kolei jest efektem tego, że Python jest językiem interpretowanym. Oznacza to, że wszędzie tam 
    gdzie istnieje interpreter tego języka, jestem w stanie uruchomić przygotowany przeze mnie kod
    (po wcześniejszym zainstalowaniu wymaganych bibliotek).

  \section{Edytor tekstu}
    W przypadku edytora tekstu zdecydowałem się na użycie programu 
    Visual Studio Code. Jest to darmowy program o otwartym kodzie źródłowym, który poza edycją tekstu 
    pozwala także na przeprowadzanie procesu debugging-u (procesu znajdowania i rozwiązywania błędów w kodzie),
    tworzenie zadań (zestaw instrukcji wykonywanych podczas egzekucji zadania) czy wspomaganie pisania kodu 
    (kolorowanie składni, szablony często używanych struktur, wskazywanie potencjalnych błędów).
    Dodatkowo ten edytor tekstu pozwala na instalację wtyczek stworzonych przez społeczność, 
    co pozwala na dalsze ułatwienie pisania kodu. Jak można dostrzec w tabeli \ref{srodowiskoProcenty},
    Visual Studio Code jest aktualnie najbardziej popularnym środowiskiem tworzenia oprogramowania, 
    dzięki czemu istnieje bardzo duża baza wiedzy na temat problemów jakie można napotkać podczas
    korzystania z tego programu.
    \begin{figure}[!htbp]
      \begin{center}
        \centering
          \begin{tabular}{|c|c|}
            \hline
            Nazwa Programu & Procent Deklarujących \\ \hline
            Visual Studio Code & 50.7\% \\ \hline
            Visual Studio & 31.5\% \\ \hline
            Notepad++ & 30.5\% \\ \hline
            IntelliJ & 25.4\% \\ \hline
            Vim & 25.4\% \\ \hline
            Sublime Text & 23.4\% \\ \hline
            Android Studio & 16.9\% \\ \hline
            Eclipse & 14.4\% \\ \hline
            PyCharm & 13.4\% \\ \hline
            Atom & 13.3\% \\
            \hline
          \end{tabular}
      \end{center}
      \caption{Procent developerów deklarujących korzystanie z danego środowiska w 2019r. \cite{StackInsight}}
      \label{srodowiskoProcenty}
    \end{figure}


\chapter{Komponenty algorytmu}

  \section{Analiza danych}
    \subsection{Opis}
      Analiza danych jest odpowiedzialna za przetworzenie wejściowego 
      modelu na postać gotową do użycia przez komponenty wykonujące dalsze obliczenia.
      Podczas tworzenia tego komponentu zdecydowałem się użyć formatu STL jako obiektu wejściowego.
      Jest to sprawdzony format zapisu danych obecny od 1987 roku. 
      Jego prosta konstrukcja (zarówno zapis binarny jak i ASCII)
      pozwala na szybkie zrozumienie zawartości pliku (w przypadku wersji ASCII), 
      co pozwala na łatwe oraz szybkie wyszukiwanie błędów
      podczas tworzenia mechanizmu wczytywania. Dodatkowo format ten jest 
      szeroko wspierany przez społeczność zajmującą się drukiem 3D,
      co objawia się obfitą ilością gotowych modeli, które w przyszłości mogą 
      zostać użyte do analizy popularnych obwodów.\\
      Jako wyjściowy format danych do dalszej pracy zdecydowałem się na mapę wysokości.
      Charakterystyka analizy badanych przedmiotów skupia się na rzutowaniu prostokątnym
      (dokładniej rzut z góry). Dzięki tej informacji mogłem stwierdzić, 
      że dane które niosą ze sobą wartość merytoryczną dla wyliczania ścieżki,
      to wysokości obiektu dla odpowiadających im punktów na płaszczyźnie poziomej.
      Efektem takiej analizy są dane, które w łatwy sposób mogą być dalej używane oraz 
      zajmują mniej pamięci. Dodatkowo w naturalny sposób opisują one obraz obiektu 
      z punktu widzenia sondy, co pozwala na łatwe stworzenie wizualizacji procesu tworzenia ścieżki.
    
    \subsection{Lista kroków}
      \begin{itemize}
        \item ładowanie modelu 3D
        \item generowanie poligonów
        \item wyliczanie płaszczyzn dla poligonów
        \item wyliczanie rozmiaru badanego obiektu
        \item tworzenie klastrów
        \item przypisywanie poligonów do klastrów
        \item wyliczanie mapy wysokości (przy zdefiniowanej rozdzielczości, dla każdego punktu w przestrzeni należy wyliczyć jego wysokość)
        \item normalizacja mapy wysokości
      \end{itemize}























































\begin{thebibliography}{99}
\addcontentsline{toc}{chapter}{Bibliografia}
\bibitem{StackInsight}{https://insights.stackoverflow.com/survey/2019}

\end{thebibliography}

\zakonczenie

\end{document}